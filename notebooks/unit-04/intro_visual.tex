\documentclass[letterpaper,11pt]{article}

\usepackage[shortlabels]{enumitem}
\usepackage[margin=1in]{geometry}
\usepackage[most]{tcolorbox}

\begin{document}
\title{{\bf Unit 4: Intro to the Visual System Project} }
\author{Name: Siddharth Thiagarajan}

\date{}
\maketitle

Imagine you got a little too into neural networks and decided to replace your eyes with convolutional neural networks. You may use any sensors, hardware, brain computer interfaces, fungi, wires, Von Neumann computing, neuromorphic computing, or robots that you like (that seem vaguely feasible). How would you replace the algorithms run by the visual cortex with algorithms like those of convolution neural networks? 

\section*{Short Answer}
\begin{enumerate}[a)]
\item Why did you make the design decisions you made?

\begin{tcolorbox}
TODO: The idea of replacing my eyes with convolutional neural networks (CNNs) might be a bit too extreme, even for an avid machine learning enthusiast like myself. Nonetheless, let's explore the possibilities hypothetically. First and foremost, it's important to note that the visual cortex is an incredibly complex and sophisticated system that performs a wide range of tasks, including edge detection, object recognition, and scene understanding. While CNNs have shown remarkable success in computer vision tasks, directly replacing the algorithms run by the visual cortex with CNNs would be challenging due to the intricate nature of the biological visual system. However, we can draw inspiration from CNNs to create an artificial visual system that emulates certain aspects of the visual cortex. To replace the algorithms of the visual cortex with CNNs, I would consider the following design decisions: Sensor Integration, CNN Architecture, Brain-Computer Interface, Feedback Mechanisms, Training and Learning. These design decisions aim to emulate the fundamental principles of CNNs while considering the limitations and intricacies of the human visual system. While this hypothetical scenario allows for creative exploration, it's essential to remember that the complexity and sophistication of the human visual system extend far beyond the current capabilities of artificial systems. The interplay of biological neurons, sensory organs, and cognitive processes creates a depth of understanding that is still unparalleled in the field of artificial intelligence. Nonetheless, the pursuit of understanding and emulating these processes is an exciting area of research with great potential for future advancements in AI and NLP.
\end{tcolorbox}

\item What would be the advantages of your system? 

\begin{tcolorbox}
TODO: Enhanced Perception, Efficient Feature Extraction, Integration with AI systems, Accessibility, Adaptability & Learning. It's important to note that these advantages are speculative and depend on the successful development and integration of such a complex artificial visual system. The human visual system is a marvel of nature, and replicating its full capabilities is an enormous challenge. Nonetheless, exploring the possibilities of merging neural networks, computer vision, and brain-computer interfaces can lead to exciting advancements in the field of AI and open up new avenues for research and innovation. 
\end{tcolorbox}

\item What would be the disadvantages?

\begin{tcolorbox}
TODO: The human visual system is incredibly complex, involving intricate connections between neurons, complex processing mechanisms, and the integration of sensory information with higher-level cognitive processes. Uncertainty in the long-term, technological limitations and cost are some other disadvantages as well. Considering these disadvantages and challenges, it is essential to approach the integration of artificial visual systems with caution, considering the potential risks and ethical implications.
\end{tcolorbox}

\item What hardware did you use to implement this? In your opinion, is it possible to use the existing biological nervous system to run computation algorithms like CNNs? Why?

\begin{tcolorbox}
TODO: I used Camera sensors, Computing Hardware (high performance CPUs/GPUs), Specialized hardware to facilitate bidirectional communication between the neural network and the brain, Neuromorphing Computing Systems as the Neural Network Hardware.
While it may be possible to integrate artificial neural networks with the biological nervous system using brain-computer interfaces, there are significant challenges and limitations. The biological nervous system is not optimized for running specific algorithms like CNNs. It is a complex network evolved for a wide range of functions beyond vision, and its computational capabilities and architecture differ from artificial neural networks. The biological nervous system has unique sensory modalities, such as touch, proprioception, and chemical senses, that contribute to our overall perception and understanding of the world. Replacing or overriding the natural sensory input with artificial visual systems would ignore the importance of these multimodal inputs and potentially limit the holistic perception of the environment.
\end{tcolorbox}
    
\end{enumerate}
\end{document}
